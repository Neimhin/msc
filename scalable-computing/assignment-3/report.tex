%% LyX 2.3.6 created this file.  For more info, see http://www.lyx.org/.
%% Do not edit unless you really know what you are doing.
\documentclass[10pt]{article}
\usepackage{helvet}
\renewcommand{\familydefault}{\sfdefault}
\usepackage[T1]{fontenc}
\usepackage[utf8]{inputenc}
\usepackage[a4paper]{geometry}
\geometry{verbose,tmargin=2cm,bmargin=4cm,lmargin=2cm,rmargin=2cm}
\usepackage{fancyhdr}
\pagestyle{fancy}
\setlength{\parskip}{6pt}
\setlength{\parindent}{0pt}
\usepackage{tcolorbox}
\usepackage{amsmath}
\usepackage{amsthm}
\usepackage{amssymb}
\usepackage{todonotes}
\usepackage{graphicx}
\usepackage[backend=biber,maxbibnames=99]{biblatex}
\addbibresource{main.bib}

\makeatletter

%%%%%%%%%%%%%%%%%%%%%%%%%%%%%% LyX specific LaTeX commands.
%% A simple dot to overcome graphicx limitations
\newcommand{\lyxdot}{.}


%%%%%%%%%%%%%%%%%%%%%%%%%%%%%% User specified LaTeX commands.
\usepackage{tcolorbox}
\usepackage{amsthm}
\usepackage{lastpage}
\usepackage{fancyhdr}
\usepackage{accents}
\usepackage{titlesec}
\usepackage{marginnote}
\usepackage{titlesec}
\titleformat{\section}[block]{\normalfont\bfseries}{}{0em}{}


\usepackage{enumitem}
\usepackage{comment}
\setlist{nolistsep}

\usepackage{tcolorbox}
\definecolor{light-blue}{cmyk}{0.24, 0.12, 0.0, 0.04, 1.00}


\newcommand{\ket}[1]{|#1\rangle}
\newcounter{kl}
\setcounter{kl}{0}
\newcommand{\kl}{
    \stepcounter{kl}
    \textbf{kl-\thekl}:
}
\newcounter{kti}
\setcounter{kti}{0}
\newcommand{\kti}{
    \stepcounter{kti}
    \textbf{kti-\thekti}:
}
\titlespacing\section{0pt}{0pt}{0pt}
\titlespacing\subsection{0pt}{0pt}{0pt}
\titlespacing\subsubsection{0pt}{0pt}{0pt}

\setlength{\headheight}{40pt}

\makeatother

\begin{document}
\lhead{Neimhin Robinson Gunning} \rhead{CS7NS1 Assignment 3} 

\bigskip{}

\section{Key Technical Insights}

\kti The entropy of an information source,
representing the expected value of its Shannon information content,
directly correlates to the efficiency of optimal compression systems.
As uncompressed string length increases,
the average bits required for encoding approaches the source's entropy,
under the assumption of encoding individual symbols.
As data and information systems grow,
ensuring security without compromising scalability gets more difficult.
QKD offers a pathway to enhance rather than sacrifice security
in tandem with the expansion
of data networks and communication systems.

\kti Slepian-Wolf coding reveals
that compressing two correlated information
sources can be optimized by using
information from one source to reduce the encoding bits needed for the other.

\kti A Nash equilibrium represents each player's best response
to the given set of strategies of the other players.
It does not necessarily correspond to the optimal outcome for all players involved
-- it is a state of mutual best response, not necessarily a state of optimal collective payoff.

\kti Unconditionally secure encryption,
exemplified by the one-time pad,
offers rigorous security but is impractical for widespread use due to key distribution challenges and single-use limitations.
Conversely, computational security, though less theoretically robust, is more feasible in common applications.
Quantum Key Distribution (QKD) could make unconditionally secure methods like the one-time pad more practical.

\begin{comment}
\kti The maximum entropy principle has gives us a decision criterion for selecting
a probability distribution when the true distribution is unknown,
namely that we should choose
the distribution such that the entropy w.r.t. any known data is maximized.
Harremoës and Topsøe \cite[23]{harremoës-2008} introduce the idea in an interesting way;
by constructing a game in which the optimal strategy
for one of the players (Nature) turns out equivalent
to applying the maximum entropy principle.
The theatrical discussion \cite[24-25]{harremoës-2008}
between the statistician and the information theorist
has the information theorist implicitly applying the maximum
entropy principle to select a probability distribution that maximizes entropy
and hence {\em minimizes risk} of a high `cost', given all that is known.
For me these two examples highlighted
the dialectic between {\em discovery} and {\em invention} in mathematics.
\end{comment}

\kti The maximum entropy principle,
which advises selecting the distribution
with maximized entropy based on known data,
offers a decision-making framework in uncertain scenarios.
Harremoës and Topsøe \cite[23]{harremoës-2008} introduce the idea in an interesting way;
by constructing a game in which the optimal strategy
for one of the players (Nature) turns out equivalent
to applying the maximum entropy principle.
For me, the example highlighted
the dialectic between {\em discovery}
and {\em invention} in mathematics.

\begin{comment}
\kti The no-cloning theorem can be derived from
requirement that quantum gates must be reversible,
and it tells us that
"quantum information cannot be created, copied, or destroyed".
The no-cloning theorem makes the classical repetition code
(an encoding that facilitates error correction and/or detection)
impossible for quantum information.
Further, since quantum information is both
susceptible to errors
and liable to leak,
the difficulty of error correction is one of the
major obstacles in the development of useful quantum computation.
\end{comment}

\kti The no-cloning theorem, a result of quantum gate reversibility,
dictates that quantum information cannot be copied or destroyed.
This prevents a simple transfer of classical error correction methods,
posing a significant hurdle in quantum computation development.
\section{Key Learnings}

\begin{comment}
\kl The linking identity, Equation (1.9) in \cite{harremoës-2008}, relates the actual average code length when 
a code $\kappa$ adapted to $Q$,
is used to encode a source with distribution
$P$ to two important quantities; 1. The entropy of the true distribution $P$ of the source being encoded, and 2. 
The divergence of true distribution $P$ from the assumed distribution $Q$ to which the id-code is adapted, 
$D(P||Q)$. While the text \cite{harremoës-2008} derives this identity from other assumptions,
I find the linking identity to be a more intuitive definition of divergence.
\end{comment}

\kl The linking identity \cite[8]{harremoës-2008} gives the relationship between average code length,
source entropy, and the divergence from assumed to true distribution,
and for me,
it offers the most intuitive understanding of KL-divergence.

\begin{comment}
\kl Bits can be encoded as qubits, but qubits cannot be encoded as bits,
presenting a challenge to the field of information theory where the bit is the fundamental unit.
However, it can still be argued that the bit remains a valid fundamental unit in information theory.
This is because a qubit's superposition state,
which cannot be encoded in bits,
also cannot be queried or observed directly.
In essence, any observation we make about a quantum system ultimately collapses to a classical state,
resulting in information that is expressed in bits.
Therefore, even though qubits represent a more complex state of information theoretically,
any information we can effectively obtain and communicate about the system is in the form of bits.
This reality reinforces philosophically the bit as the basic unit of information epistemologically,
even if the qubit does introduce a richer ontological structure.
\end{comment}

\kl While bits can be encoded in qubits,
the inverse isn't true,
presenting challenges in information theory.
However, since observations of quantum systems collapse to classical states,
bits remain the fundamental information unit epistemologically,
despite the richer ontological structure of qubits.

\begin{comment}
\kl Quantum computation has similarities to analog computation,
such as the Difference Engine,
or the way multiplication of continuous
numbers can be encoded as the addition of two voltages in a circuit.
When considering quantum computation as an
actual tool for completing tasks
it is important to remember that some advantages of quantum computation
can already be offered by classical
devices whose properties are
easier to control.
The great disadvantage of such devices is the inability to
copy the information perfectly, whereas digital information can
be copied and disseminated perfectly with an arbitrary degree of confidence.
Work on achieving similar guarantees for continuous and quantum information is ongoing,
but the challenge may be insurmountable.
\end{comment}

\begin{comment}
\kl Digital information makes sense in an entirely logical framework, i.e. detached from
physical reality, but at the same time the tools of information theory allow us
to create physical manifestations of digital information, in storage systems, in RF or acoustic signals,
and then to protect that digital information to an arbitrary degree.
I think the abstractness of digital information has been fundamental to enabling
digital to scale to the point of eating the world.
Harremoës and Topsøe claim that in "quantum information theory we have to perform a measurement before coding",
i.e. are hands are tied more strictly to the physical reality.
I think the lesson here is that quantum information theory provides {\em optimizations} for specific tasks,
but digital information, being a touch more abstract, a touch less shackled to physical reality,
has in fact led to it, thus far, being the weapon of choice for practical problems.
\end{comment}

\kl Digital information's abstraction from physical reality
has been key to its scalability and dominance.
Quantum information theory, however, is more intimately tied to physical realities.
While quantum information theory can optimize specific tasks,
digital information remains more practical for a broader range of applications.
This contrast between digital and quantum information
highlights a fundamental scalability trade-off:
the ease and flexibility of scaling digital information systems
versus the physical tethering of quantum systems.

\kl We can never achieve perfect certainty that a message is received
perfectly when sent through a channel, but we can engineer systems
to guarantee arbitrarily high certainty depending on how much we are
willing to sacrifice in latency, throughput, money etc.

\begin{comment}
\kl We can already construct coding schemes that give maximum capacity
with negligible error in the limit.
Shannon constructed one to prove the second main theorem of information theory.
But this is only in the limit,
i.e. if we are going to send an infinitely long message,
which I have never had any reason to do!
One goal of modern coding scheme development,
then, is to construct schemes that approach the capacity for practically short messages.
However, another line of research seeks to construct coding schemes with tolerably non-negligible errors,
but potentially higher capacity.
\end{comment}

\kl Achieving maximum coding capacity with negligible error
is only theoretically possible for infinitely long messages.
Current research focuses on optimizing coding schemes
for practical message lengths.
However, another line of research seeks to construct coding schemes
with tolerably non-negligible errors,
but potentially higher capacity.

\kl I think we are gradually gaining insight into compression as a form or requisite of intelligence
(I’m thinking of GPTs as fancy text compressors which then miraculously mimic intelligence).
Synthesizing this with the thought "data compression
emphasizes structure" \cite[38]{harremoës-2008} we might benefit
from looking at intelligence as the thing that identifies and encodes
structure while intelligently discarding unstructure.


\section{Quantum Information and Computation since 2008}
A major challenge in practical application of quantum information theory
is the instability of quantum systems, which means, for instance,
that it is
very difficult to store a qubit for a long period of time.
One of the avenues of research that is providing
solutions to this problem is in Quantum Error Correction.
As with classical bits, quantum bits are
in practice susceptible to bit flips,
e.g. if the state of a qubit changes
erroneously from \( \ket{0} \) to \( \ket{1} \).
However, qubits are additionally susceptible to phase errors,
e.g. a state like \( \alpha\ket{0} + \beta\ket{1} \) changing to
\( \alpha\ket{0} - \beta\ket{1} \).

There are very fresh reports that a team at AWS have improved
quantum error correction 100-fold by separating out the
different types of
qubit flips and tackling them individually \cite{swayne-2023},
but a technical publication is yet to emerge.

\begin{comment}
The Deutsch-Jozsa \cite{deutsch-1992} algorithm was one of the first
theoretical demonstrations of a quantum algorithm which
can, in theory, solve specific problems exponentially faster than
the best know contemporaneous classical algorithms.
However, the Deutsch-Jozsa algorithm, like many other quantum algorithms,
has requirements for coherent evolution of the system that prove
very difficult to achieve in practice.
\end{comment}

Shor's prime factorization algorithm was published in 1997,
and the first successful implementation only 4 years later in 2001 \cite{vandersypen-2001},
and yet the algorithm still has not been applied
to gain a practical speed increase compared to classical computers,
so naturally a great deal of the energy and money
going to into quantum research has been
on incremental improvements to implementations.

\begin{comment}
Unlike classical information,
which was born fully fledged in Shannon's 1948 paper,
the theoretical aspect of quantum computation is anchored to
a much greater extent in empirical investigations and experimentation.
Quantum information is realised in all physical systems,
which is it why it is so attractive,
but it means the theory doesn't progress independently of the empirics.
I think quantum technological development is characterized by such
double-edged swords, e.g. the no-cloning theorem
is necessary for the security of quantum key distribution,
but poses a huge challenge to reliable communication
of quantum information in general.
\end{comment}

There are two main approaches to making progress on practical applications
of quantum systems to information and computation problems:
1. To design algorithms with lower coherence requirements,
2. To design physical devices with better coherence guarantees.
Progress on the algorithms side in relation to the
variational eigenvalue solving was published in 2014 \cite{peruzzo-2014}.
The technique has much lower requirements for coherence than
the quantum phase estimation algorithm,
and therefore made some computations newly feasible.
However, satelite-based entanglement distribution \cite{yin-2017},
a landmark technology introduced in 2017,
drastically increased practical coherence distance without a big increase in coherence time.

\begin{comment}
Quantum Secure Direct Communication (QSDC) \cite{long-2017},
unlike QKD, is a quantum cryptography
technique that does not require the distribution of a shared secret key.
To quote Long's salacious emphasis ``QSDC has become one hot research area'' in the decade following the 2008 publication of the chapter
by Harremoës and Topsøe \cite{harremoës-2008}.
\end{comment}

Harremoës and Topsøe \cite[38]{harremoës-2008} noted that
the {\em additivity conjecture} remained unproven in 2008,
but that same year Smith and Yard \cite{smith-2008} published a counterexample to the conjecture,
two quantum channels each with zero capacity, but when combined have non-zero capacity.
The unsettling implication is that the capacity of a quantum channel does not fully describe
its ability to transmit information.

\section{Quantum Information and Computation from now until 2028}

We inch closer to quantum engineering capabilities
that we know theoretically
will be highly disruptive.
The trend lines of quantum coherence guarantees
and quantum algorithm requisites are steadily converging.
But, I think there is also a good possibility of
new breakthrough technologies emerging more quickly out of the blue.
The key to quantum information and computation is
in leveraging its fundamental oddities,
and I think there is still a great deal of headroom
in exploring the truly novel capabilities quantum information
might facilitate when we unshackle ourselves from classical idioms.
The discovery that certain quantum channels are {\em non-additive}
is an example of the kind of out-of-the-blue quantum weirdness
that might lead to equally surprising practical applications in the coming years.

The classical computation framework already serves us very well
for tackling almost any problem, albeit at limited scales,
and so researchers in quantum will focus on tasks where
the quantity, the scale, has a quality of its own.


\setlength\bibitemsep{0.3\itemsep}
\renewcommand{\bibfont}{\footnotesize}
\printbibliography
\end{document}
